\documentclass[
	a4paper,
	% showframes,
	% vline=2.2em,
	maincolor=cvblue!70!blue,
	sidecolor=gray!30,
	sectioncolor=cvblue!70!blue,
    sidebarwidth=7.5cm,
	topbottommargin=20pt,
	leftrightmargin=20pt,
]{fortysecondscv}
%
\usepackage[spanish]{babel}
\hyphenation{Toapanta}
% Espaciado 
\usepackage{microtype}
\usepackage{ragged2e}
\usepackage{silence}
\WarningFilter{latex}{Overfull \hbox}
\WarningFilter{latex}{Underfull \hbox}
\WarningFilter{latex}{Unused global option(s):}


% Fuente Hola
\usepackage{fontspec}
\setmainfont{montserrat}


% -- Información personal
\cvname{Andrés Esteban\\[3mm] Merino Toapanta}
\cvjobtitle{Matemático\\[2mm] Docente-Investigador}
\cvprofilepic{pics/aemerinot.jpg}
\cvbirthday{18 de octubre de 1990}
\cvaddress{Mariscal Sucre, Quito - Ecuador}
\cvphone{(+593) 98 975 1881}
\cvmail{aemerinot@gmail.com}
\cvmail{aemerinot@puce.edu.ec}
\cvcustomdata{\faFlag}{Ecuatoriano}

% -- Primera página
\addtofrontsidebar{
	\graphicspath{{pics/flags/}}

	% \profilesection{Perfiles en línea}
	% 		\social{\aiOrcidSquare}
	% 			{https://orcid.org/0000-0002-5404-918X}
	% 			{Orcid}
	% 		\social{\faGithub}
	% 			{https://github.com/andres-merino}
	% 			{Github}
	% 		\social{\faLinkedin}
	% 			{https://www.linkedin.com/in/andres-merino-t}
	% 			{LinkedIn}

	\profilesection{Idiomas}
		\pointskill{\flag{ES.png}}{Español}{5}
    	\pointskill{\flag{GB.png}}{English}{4}
    	\pointskill{\flag{FR.png}}{Française}{2}

	\profilesection{Áreas de interés}
        \skill{\faRebel}{Teoría descriptiva de conjuntos}
        \skill{\faRebel}{Fundamentos de la Matemática}
        \skill{\faRebel}{Ciencia de datos}
        \skill{\faRebel}{Aprendizaje automático}
        \skill{\faRebel}{Edición y composición en \LaTeX}
        \skill{\faRebel}{Educación matemática}
}


% -- Segunda página
\addtobacksidebar{
	\profilesection{Sobre mi}
	\aboutme{
		Ha dedicado mi carrera a la docencia y la investigación. En este último, colabor en las áreas de la Teoría Descriptiva de Conjuntos, Fundamentos de la Matemática, Educación Matemática y Ciencia de Datos. He aportado conocimientos y avances mediante publicaciones en revistas científicas y participación en congresos nacionales e internacionales. Mis contribuciones incluyen estudios sobre la Derivada de Cantor-Bendixson, Modelos de aprendizaje automático para análisis de datos e imágenes y Técnicas innovadoras para la enseñanza de la Matemática.
	}

	\profilesection{Software}
    	\barskill{\faPenFancy}{LaTeX}{100}
    	\barskill{\faPencilRuler}{PSTricks}{100}
    	\barskill{\faRuler}{GeoGebra}{90}
    	\barskill{\faPython}{Phyton}{80}
    	\barskill{\faCalculator}{Mathematica}{60}
    	\barskill{\faFileImage}{Inkscape}{60}
    	\barskill{\faRProject}{R}{50}
    	\barskill{\faDatabase}{SQL}{40}
    	\barskill{\faGit*}{Git}{30}
    	\barskill{\faPencilRuler}{TikZ}{30}
}

% -- Documento
\begin{document}

\makefrontsidebar

%%%%%%%%%%%%%%%%%%%%%%%%%%%%%%
\cvsection{Educación}
%%%%%%%%%%%%%%%%%%%%%%%%%%%%%%
\cvitem{feb-2024}
    {Máster Universitario en Ciencia de Datos}
    {Universitat Oberta de Catalunya}
    {\hspace{0pt}\\[-1mm] Trabajo de titulación: Reconocimiento automático de imágenes para reconstrucción de redes planta-polinizador\\[-1.5mm]}

\cvitem{mar-2017}
    {Magíster en Matemáticas Puras y Aplicadas}
    {Universidad Central del Ecuador}
    {\hspace{0pt}\\[-1mm] Trabajo de titulación: Clasificación de Subconjuntos Compactos Numerables en algunos espacios Polacos}

\cvitem{oct-2014}
    {Master 1}
    {Université Jean Monnet}
    {Master Mathematiques 1}
    
\cvitem{oct-2014}
    {Matemático}
    {Escuela Politécnica Nacional}
    {Trabajo de titulación: Clasificación de Subconjuntos Compactos Numerables de los Reales\\ \textit{Summa Cum Laud}}
    
\cvitem{jul-2008}
    {Bachiller}
    {Colegio Municipal Experimental ``Sebastián de Benalcázar''}
    {Especialidad en Físico Matemático}

%%%%%%%%%%%%%%%%%%%%%%%%%%%%%%
\cvsection{Experiencia laboral docente}
%%%%%%%%%%%%%%%%%%%%%%%%%%%%%%
\cvitem{abr-2017}
    {Profesor Agregado I}
    {Pontificia Universidad Católica del Ecuador}
    {Profesor de Matemática en la Facultad de Ciencias Exactas y Naturales (trabajo actual a tiempo completo: 8 años)}
    
\cvitem{abr-2021}
    {Docente de posgrado}
    {Pontificia Universidad Católica del Ecuador}
    {Docente en la Maestría en Pedagogía de las Ciencias Experimentales, mención en Matemática y Física (a tiempo parcial:~1~año y~6 meses)}

\cvitem{mar-2018}
    {Profesor Ocasional}
    {Escuela Politécnica Nacional}
    {Profesor de Matemática en el Departamento de Formación Básica (a medio tiempo: 12 meses)}

\cvitem{oct-2014}
    {Profesor Ocasional}
    {Escuela Politécnica Nacional}
    {Profesor de Matemática en la Facultad de Ciencias (2 años y~5~meses)}

\cvitem{abr-2014}
    {Docente}
    {Colegio Municipal ``Sebastián de Benalcazár''}
    {Profesor de la cátedra de Matemática (4 meses)}

%%%%%%%%%%%%%%%%%%%%%%%%%%%%%%
\cvsection{Experiencia laboral fuera de la docencia}
%%%%%%%%%%%%%%%%%%%%%%%%%%%%%%

\cvitem{abr-2021}
    {Coordinador de posgrado}
    {Universidad Central del Ecuador}
    {Coordinador Maestría en Matemáticas Puras y Aplicadas en la Facultad de Ciencias (a tiempo parcial: 1 año y 9 meses)}

\cvitem{jun-2019}
    {Editor}
    {Ministerio de Educación}
    {Editor de módulos de Matemática para 10.o año de EGB en la Dirección Nacional de Currículo (1 meses)}

\cvitem{feb-2018}
    {Validador}
    {Instituto Nacional de Evaluación Educativa (INEVAL)}
    {Validación de producción de ítem 2018 en la Dirección de Elaboración y Resguardo de Ítem (2 meses)}

\cvitem{may-2015}
    {Monitor}
    {Instituto de Altos Estudios Nacionales (IAEN)}
    {Monitor de curso virtual de Matemática en el Centro de Educación Continua (CEC-EP) (1 mes)}

\cvitem{oct-2012}
    {Auxiliar de laboratorio}
    {Escuela Politécnica Nacional}
    {Apoyo en el proyecto de investigación semilla ``Flujo sobre ensambles granulares'' (1 año)}

\cvitem{dic-2011}
    {Asistente de edición}
    {Abya Yala, Revista La Granja}
    {Maquetación, diagramación y diseño de plantillas (3 años)}

\cvitem{mar-2011}
    {Auxiliar de laboratorio}
    {Escuela Politécnica Nacional}
    {Edición en \LaTeX y apoyo en clases de Cálculo en una variable y Fundamentos de la Matemática (1 año y 3 meses)}

\cvitem{ago-2010}
    {Asistente de edición}
    {Escuela Politécnica Nacional}
    {Edición y publicación de libros y revistas en la Unidad de publicaciones de la Facultad de Ciencias (6 meses)}

\cvitem{nov-2010}
    {Asistente de edición}
    {Analítika, revista de análisis estadístico}
    {Edición, diagramación y maquetación. (3 años)}


\makebacksidebar

%%%%%%%%%%%%%%%%%%%%%%%%%%%%%%
\cvsection{Designaciones}
%%%%%%%%%%%%%%%%%%%%%%%%%%%%%%

\cventry{mar-2025}
    {Responsable de la Unidad de Servicio para el Desarrollo e Innovación Curricular}
    {Pontificia Universidad Católica del Ecuador, Facultad de Ciencias Exactas, Naturales y Ambientales}
    {Designación actual}

% \cventry{jul-2024}
%     {Vocal suplente del Consejo de Facultad de Ciencias Exactas y Naturales}
%     {Pontificia Universidad Católica del Ecuador, Facultad de Ciencias Exactas y Naturales}
%     {Designación actual}

\cventry{jul-2023}
    {Coordinador de la Escuela de Ciencias Físicas y Matemática}
    {Pontificia Universidad Católica del Ecuador, Facultad de Ciencias Exactas y Naturales, Escuela de Ciencias Físicas y Matemáticas}
    {1 año y 8 meses}
    
\cventry{mar-2020}
    {Coordinador de aulas virtuales}
    {Pontificia Universidad Católica del Ecuador, Facultad de Ciencias Exactas y Naturales, Escuela de Ciencias Físicas y Matemáticas}
    {3 años}
    
\cventry{abr-2021}
    {Coordinador Maestría en Matemáticas Puras y Aplicadas}
    {Universidad Central del Ecuador, Facultad de Ciencias}
    {1 año y 9 meses}
    
\cventry{sep-2018}
    {Coordinador de la cátedra de Cálculo Vectorial}
    {Escuela Politécnica Nacional, Departamento de Formación Básica}
    {6 meses}
    
\cventry{sep-2018}
    {Coordinador de la cátedra de Álgebra Lineal}
    {Escuela Politécnica Nacional, Departamento de Formación Básica}
    {6 meses}
    

%%%%%%%%%%%%%%%%%%%%%%%%%%%%%%
\cvsection{Cursos universitarios dictados}
%%%%%%%%%%%%%%%%%%%%%%%%%%%%%%

\cvitemshort{}{Aprendizaje Automático}
\cvitemshort{}{Cálculo en una variable}
\cvitemshort{}{Cálculo vectorial}
\cvitemshort{}{Análisis Matemático}
\cvitemshort{}{Lógica Matemática y Teoría de Conjuntos}
\cvitemshort{}{Análisis Real}
\cvitemshort{}{Fundamentos de Matemática}
\cvitemshort{}{Ecuaciones Diferenciales Ordinarias}
\cvitemshort{}{Álgebra Lineal}
\cvitemshort{}{Lógica Difusa}
\cvitemshort{}{Matemáticas discretas}
\cvitemshort{}{Matemática para Diseño Gráfico}

\newpage
\newgeometry{
	top=2cm,
	bottom=2cm,
	right=2cm,
	left=2cm
}



%%%%%%%%%%%%%%%%%%%%%%%%%%%%%%
\cvsection{Cursos y conferencias dictados  (últimos 5 años)}
%%%%%%%%%%%%%%%%%%%%%%%%%%%%%%
\cventry{jun-2025}
    {Conferencia: Exploración del potencial didáctico de las alucinaciones del ChatGPT en la enseñanza}
    {Workshop Internacional INCOIN LEARNING: Inteligencia Artificial aplicada a la Educación Superior, edición PUCE 2025}
    {Pontificia Universidad Católica del Ecuador}
    
\cventry{ago-2024}
    {Conferencia: Integración de ChatGPT en la Metodología de Aula Invertida}
    {Workshop INCOIN: Inteligencia Artificial en la Educación Superior, edición PUCE 2024}
    {Pontificia Universidad Católica del Ecuador}
    
\cventry{feb-2024}
    {Ponencia: Uso de ChatGPT para la enseñanza del cálculo diferencial}
    {MEM 2024}
    {Universidad Antonio Nariño}
    
\cventry{oct-2023}
    {Ponencia: ¿ChatGPT sabe derivar?, una experiencia en el aula}
    {Seminario STEM Miami 2023}
    {Broward International University}
    
    
\cventry{ago-2023}
    {Minicurso: Aprendizaje por refuerzo}
    {Escuela de verano en Aprendizaje Automática 2023}
    {Pontificia Universidad Católica del Ecuador (6 horas)}
    
    
\cventry{jul-2023}
    {Panel: Coordinadores y coordinadoras de  programas académicos de pregrado y posgrado de Ciencia de Datos}
    {I Encuentro métodos y herramientas de Big Data e Inteligencia Artificial aplicados al estudio de fenómenos sociales en Ecuador}
    {Flacso Ecuador}
    
\cventry{sep-2022}
    {Ponencia: Guía de actividades para refuerzo de matemática basada en la gamificación dirigida al nivel básico superior}
    {II Congreso Internacional de Educación: Crisis, Creatividad y Transformación Educativa}
    {Pontificia Universidad Católica del Ecuador}
    
\cventry{feb-2022}
    {Taller: Didácticas innovadoras para la enseñanza, aprendizaje de matemáticas}
    {}
    {La Salle, Distrito Lasllista Norandino (30 horas)}
    
\cventry{nov-2021}
    {Video ponencia: Diseño de tipografías a través de curvas de bézier y GeoGebra}
    {VIII Taller Internacional “Tendencias en la Educación Matemática Basada en la Investigación en alianza con la Comunidad GeoGebra Latinoamericana”}
    {Benemérita Universidad Autónoma de Puebla}
    

\cventry{sep- 2021}
    {Curso: Manejo del Software Geogebra Nivel Medio}
    {}
    {Universidad Regional Amazónica Ikiam (10 horas)}
    
\cventry{dic-2020}
    {Conferencia: Una generalización del conjunto ternario de Cantor}
    {II Jornadas Ecuatorianas de Matemáticas}
    {Universidad Universidad Técnica de Manabí}
    

\cventry{dic-2020}
    {Conferencia: Equivalencia del Axioma de Elección con el problema de redefinición de funciones}
    {II Jornadas Ecuatorianas de Matemáticas}
    {Universidad Universidad Técnica de Manabí}
    
\cventry{dic-2020}
    {Conferencia: Superficies regladas en GeoGebra como vínculo entre la Matemática y la Arquitectura}
    {II Jornada Ecuatoriana de GeoGebra}
    {Universidad Nacional de Educación}
    

% \cventry{feb-2020}
%     {Curso: Herramientas tecnológicas para la enseñanza de la Matemática}
%     {}
%     {Pontificia Universidad Católica del Ecuador (40 horas)}
    
% \cventry{oct-2019}
%     {Curso: Desarrollo del pensamiento en base a la resolución de problemas matemáticos}
%     {}
%     {Universidad Técnica del Norte (40 horas)}
    
% \cventry{ago-2019}
%     {Conferencia: Elaboración de animaciones en GeoGebra para motivar el estudio de geometría analítica}
%     {I Jornada Ecuatoriana de GeoGebra}
%     {Universidad Nacional de Educación}
    
%%%%%%%%%%%%%%%%%%%%%%%%%%%%%%%%%%%%%%%%
\cvsection{Jornadas de conferencias asistidas (últimos 5 años)}
%%%%%%%%%%%%%%%%%%%%%%%%%%%%%%%%%%%%%%%%

    
\cventry{ago-2024}
    {Workshop INCOIN: Inteligencia Artificial en la Educación Superior, edición PUCE 2024}
    {Pontificia Universidad Católica del Ecuador}
    {Quito-Ecuador (18 horas)}
    
\cventry{oct-2023}
    {Seminario STEM Miami 2023}
    {Broward International University}
    {Miami-Estados Unidos (8 horas)}

\cventry{sep-2022}
    {II Congreso Internacional de Educación: Crisis, Creatividad y Transformación Educativa}
    {Pontificia Universidad Católica del Ecuador}
    {Quito-Ecuador (40 horas)}

\cventry{jun-2022}
    {I Simposio Internacional de Física y Matemáticas}
    {Pontificia Universidad Católica del Ecuador}
    {Quito-Ecuador (10 horas)}

\cventry{feb-2022}
    {XII Simposio de Matemática y Educación Matemática, XI Congreso Internacional de Matemática asistida por Computador (MEM2022)}
    {Universidad Antonio Nariño}
    {Bogotá-Colombia (20 horas)}

\cventry{ene-2022}
    {Congreso de Investigación Aplicada a Ciencia de Datos – II Congreso Nacional de R Users Group}
    {Escuela Politécnica Nacional}
    {Quito-Ecuador (34 horas)}

\cventry{nov-2021}
    {VIII Taller Internacional “Tendencias en la Educación Matemática Basada en la Investigación en alianza con la Comunidad GeoGebra Latinoamericana”}
    {Benemérita Universidad Autónoma de Puebla}
    {Puebla-México (10 horas)}

% \cventry{may-2019}
%     {I Jornada Ecuatoriana de GeoGebra}
%     {Universidad Nacional de Educación}
%     {Azogues-Ecuador (20 horas)}

%%%%%%%%%%%%%%%%%%%%%%%%%%%%%%%%%%%%%%%%
\cvsection{Cursos de capacitación docente  (últimos 5 años)}
%%%%%%%%%%%%%%%%%%%%%%%%%%%%%%%%%%%%%%%%

\cventry{jul-2025}
    {Aprendizaje Basado en Retos}
    {Tecnológico de Monterrey}
    {Monterrey-México (26 horas)}

\cventry{sep-2025}
    {Diplomado en Docencia Universitaria}
    {Pontificia Universidad Católica del Ecuador}
    {Quito-Ecuador (140 horas)}

\cventry{feb-2025}
    {Exploración profunda de las herramientas de Moodle}
    {Pontificia Universidad Católica del Ecuador}
    {Quito-Ecuador (40 horas)}

\cventry{ene-2025}
    {Creación de Experiencias de Aprendizaje mediadas por la IA}
    {Tecnológico de Monterrey}
    {Monterrey-México (24 horas)}

\cventry{feb-2024}
    {IA para mejorar el proceso de enseñanza y aprendizaje}
    {IDEA-USFQ}
    {Quito-Ecuador (6 horas)}

\cventry{jul-2023}
    {Inteligencia artificial aplicada a la Educación}
    {Universidad Israel}
    {Quito-Ecuador (80 horas)}

\cventry{jul-2022}
    {PUCE INNOVA: Hacia una transformación pedagógica universitaria}
    {Reimagine Education}
    {Barcelona-España (16 horas)}

\cventry{dic-2021}
    {Concepto y desarrollo de un prototipo para la transformación educativa mediante la metodología RIEDUSIS}
    {Reimagine Education}
    {Barcelona-España (65 horas)}

\cventry{oct-2021}
    {Certificación en neurociencia y educación aplicada a la docencia}
    {Pontificia Universidad Católica del Ecuador}
    {Quito-Ecuador (50 horas)}

\cventry{abr-2021}
    {Creación de actividades autoevaluables con GeoGebra}
    {Instituto GeoGebra Extremeño}
    {Mérida-España (30 horas)}

\cventry{feb-2021}
    {Curso avanzado para manejo de aulas virtuales}
    {Pontificia Universidad Católica del Ecuador}
    {Quito-Ecuador (24 horas)}

% \cventry{ago-2020}
%     {Certificación de Docencia en Ambientes Virtuales de Aprendizaje}
%     {Pontificia Universidad Católica del Ecuador}
%     {Quito-Ecuador (120 horas)}

% \cventry{may-2020}
%     {Creando ambientes de enseñanza aprendizaje con Google Classroom}
%     {Escuela Politécnica Nacional}
%     {Quito-Ecuador (48 horas)}

% \cventry{jul-2019}
%     {Moodle 3.5}
%     {Pontificia Universidad Católica del Ecuador}
%     {Quito-Ecuador (40 horas)}

%%%%%%%%%%%%%%%%%%%%%%%%%%%%%%%%%%%%%%%%
\cvsection{Otros cursos de capacitación (últimos 5 años)}
%%%%%%%%%%%%%%%%%%%%%%%%%%%%%%%%%%%%%%%%

\cventry{mar-2025}
    {Introducción al Aprendizaje por Refuerzo}
    {CEDIA}
    {Cuenca-Ecuador (40 horas)}

\cventry{feb-2025}
    {Efficient Academy Writer with IA}
    {Sociedad Ecuatoriana de Estadística}
    {Quito-Ecuador (20 horas)}

\cventry{feb-2025}
    {Política General de Seguridad de la Información}
    {Pontificia Universidad Católica del Ecuador}
    {Quito-Ecuador (20 horas)}

\cventry{nov-2024}
    {Elementos de IA}
    {University of Helsinki}
    {(50 horas)}

\cventry{nov-2024}
    {Prevención de violencia basada en género}
    {Pontificia Universidad Católica del Ecuador}
    {Quito-Ecuador (20 horas)}

\cventry{nov-2022}
    {Amazon Web Service for Data Science}
    {SDC Learning}
    {Lima-Perú (64 horas)}

\cventry{mar-2021}
    {Machine Learning con Python}
    {Universidad Peruana Cayetano Heredia}
    {Lima-Perú (32 horas)}

% \cventry{may-2020}
%     {Herramientas de Office 365}
%     {Escuela Politécnica Nacional}
%     {Quito-Ecuador (64 horas)}

% \cventry{jun-2019}
%     {Pares evaluadores PUCE}
%     {Pontificia Universidad Católica del Ecuador }
%     {Quito-Ecuador(40 horas)}

%%%%%%%%%%%%%%%%%%%%%%%%%%%%%%
\cvsection{Publicaciones}
%%%%%%%%%%%%%%%%%%%%%%%%%%%%%%

\cvpubitem{dic-2023}
    {Artículo científico}
    {Merino, A. y Heredia, S. (2023). Relación de la Derivada de Cantor-Bendixson con el Álgebra de Conjuntos. Selecciones Matemáticas, 10(2), 339-351.}
    {\url{https://doi.org/10.17268/sel.mat.2023.02.10}}
    {Área: Teoría Descriptiva de Conjuntos}
    
\cvpubitem{ene-2023}
    {Artículo científico}
    {Jiménez, S. y Merino, A. (2023). Modelos de Aprendizaje Automático basados CRISP-DM para el Análisis de los niveles de Depresión en los estudiantes de la Escuela Politécnica Nacional. Latin-American Journal of Computing, 10(1), 22-43.}     
    {\url{https://doi.org/10.5281/zenodo.7503909}}
    {Área: Ciencia de datos}
    
\cvpubitem{ago-2022}
    {Artículo científico}
    {Merino, A., Cueva, M., Sarmiento, M., y Paredes, A. (2022). Análisis de los conceptos de variable y constante en los estudiantes del Bachillerato General Unificado del Ecuador. 593 Digital Publisher CEIT, 7(4-2), 175-185.}     
    {\url{https://doi.org/10.33386/593dp.2022.4-2.970}}
    {Área: Educación matemática}
    
\cvpubitem{jun-2021}
    {Artículo científico}
    {Merino, A. y Ortiz-Castro, J. (2021). Continuidad de funciones basadas en reordenamientos de $\beta$-expansiones de un número. Revista digital Matemática, Educación e Internet, 22(1).}     
    {\url{https://doi.org/10.18845/rdmei.v22i1.5758}}
    {Área: Análisis Real}
    
\cvpubitem{feb-2021}
    {Artículo científico}
    {Álvarez-Samaniego, B. y Merino, A. (2021). Some properties related to the Cantor-Bendixson derivative on a Polish space. New Zealand Journal of Mathematics, 50, 207-218.}     {\url{http://dx.doi.org/10.17268/sel.mat.2020.02.04}}
    {Área: Teoría Descriptiva de Conjuntos}
    
\cvpubitem{dic-2020}
    {Artículo científico}
    {Merino, A. y Heredia, S. (2020). Una generalizacion del Conjunto ternario de Cantor. Selecciones Matemáticas, 7(2), 222-233.}     {\url{http://dx.doi.org/10.17268/sel.mat.2020.02.04}}
    {Área: Análisis Real}
    
\cvpubitem{may-2020}
    {Artículo científico}
    {Merino, A. y Trujillo, J. C. (2020). Equivalencia del Axioma de Elección con el Problema de Redefinición de Funciones. Revista Politécnica, 45(2), 51-56.}     {\url{https://doi.org/10.33333/rp.vol45n2.05}}
    {Área: Fundamentos de la Matemática}
    
\cvpubitem{nov-2019}
    {Artículo científico}
    {Álvarez-Samaniego, B. y Merino, A. (2019). Countable ordinal spaces and compact countable subsetes of a metric space. Australian Journal of Mathematical Analysis and Applications, 16(2), artículo 12.}
    {\url{https://ajmaa.org/searchroot/files/pdf/v16n2/v16i2p12.pdf}}
    {Área: Teoría Descriptiva de Conjuntos}
    
\cvpubitem{jun-2019}
    {Artículo científico}
    {Ortiz, J. y Merino, A. (2019). Esquematización del problema de clasificación de conjuntos. La clasificación de la ecuación cuadrática de dos variables. Matemática ESPOL-FCNM JOURNAL, 17(1), 1-14.}
    {\url{https://www.revistas.espol.edu.ec/index.php/matematica/article/view/538}}
    {Área: Educación matemática}
    
\cvpubitem{sep-2016}
    {Artículo científico}
    {Álvarez-Samaniego, B. y Merino, A. (2019). A primitive associated to the Cantor-Bendixson derivative on the real line. Journal of Mathematical Sciences: Advances and Applications 41 (1).}     
    {\url{http://dx.doi.org/10.18642/jmsaa_7100121692}}
    {Área: Teoría Descriptiva de Conjuntos}
    
\cvpubitem{jun-2012}
    {Folleto}
    {Merino, A., Cueva, E. y Trujillo J.C. (2012). Cálculo en una variable: resumen y ejercicios resueltos. Unidad de Publicaciones de la Facultad de Ciencias.}
    {}
    {Área: Educación matemática}
    


%%%%%%%%%%%%%%%%%%%%%%%%%%%%%%
\cvsection{Galardones, reconocimientos y becas}
%%%%%%%%%%%%%%%%%%%%%%%%%%%%%%
\cventry{feb-2025}
    {Reconocimiento por el alto desempeño en la práctica docente}
    {Facultad de Ciencias Exactas, Naturales y Ambientales - PUCE}
    {}

\cventry{dic-2023}
    {Botón de gratitud}
    {Asociación de Profesores de la Pontificia Universidad Católica del Ecuador}
    {}

\cventry{jul-2021}
    {Beca UOC para estudiantes sobresalientes de Ecuador}
    {Universitat Oberta de Catalunya}
    {}
    
\cventry{dic-2017}
    {Premio Municipal Pedro Vicente Maldonado a la mejor obra publicada en el área de ciencias exactas del año 2017}
    {Municipio del Distrito Metropolitano de Quito}
    {Obra: A primitive associated to the Cantor Bendixson derivate in the real line}
    

\cvsignature{14 de noviembre de 2025}

\end{document}

Andrés Esteban Merino Toapanta es Profesor Agregado I en la Pontificia Universidad Católica del Ecuador y Responsable de la Unidad de Servicio para el Desarrollo e Innovación Curricular. Su formación académica integra el título de Matemático de la Escuela Politécnica Nacional con distinción Summa Cum Laude, y es ampliada a través de una Maestría en Matemáticas Puras y Aplicadas por la Universidad Central del Ecuador y un Máster Universitario de Ciencia de Datos por la Universitat Oberta de Catalunya.

Ha dedicado su carrera a la docencia y la investigación. En este último, colabora en las áreas de la Teoría Descriptiva de Conjuntos, Fundamentos de la Matemática, Educación Matemática y Ciencia de Datos. Ha aportado conocimientos y avances mediante publicaciones en revistas científicas y participación en congresos nacionales e internacionales. Sus contribuciones incluyen estudios sobre la Derivada de Cantor-Bendixson, Modelos de aprendizaje automático para análisis de datos e imágenes y Técnicas innovadoras para la enseñanza de la Matemática.
